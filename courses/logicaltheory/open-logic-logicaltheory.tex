\documentclass[a4paper]{memoir}



\newcommand{\olpath}{../../}
\input{\olpath/sty/open-logic.sty}
\input{\olpath/sty/open-logic-defer.sty}
\makeatletter\AtBeginDocument{\let\@elt\relax}\makeatother

\problemsperchapter

\hypersetup{%pdftex,
%  breaklinks,
%  bookmarks,
%  bookmarksopen,
%  bookmarksopenlevel=1,
  hidelinks}
  %colorlinks,
  %urlcolor=dark-gray}
  %linkcolor=reflex-blue}

\gdef\proofSymbol{\ensuremath{\Box}}


%\usepackage{enumitem}
%\setlist{topsep=6pt,itemsep=0pt,parsep=2pt,label={--}} % or \setlist{noitemsep} to leave space around whole list


\usepackage[lining,proportional]{libertine}
\usepackage[libertine]{newtxmath}

\usepackage{microtype}

%\usepackage{mathpazo}
%\usepackage[scaled=0.95]{helvet}
%\linespread{1.05}

% e-book-friendly version:
%\usepackage[papersize={3.6in,4.8in},hmargin=0.1in,vmargin={0.1in,0.1in}]{geometry}

%\sloppy
\setlength{\emergencystretch}{4pt}

\begin{document}

% First we make a titlepage

\begin{titlingpage}
\begin{raggedleft}
\fontsize{52pt}{2em}\selectfont\bfseries\sffamily
Logical\\[.5ex] 
Theory
\vskip 4ex
%\normalfont\Huge\textbf{\href{http://openlogicproject.org/}{Open Logic Project}}
%\vskip 1ex
\Large\today

\end{raggedleft}

\vfill

\begin{minipage}[b]{.9cm}
\includegraphics[width=.9cm]{\olpath/assets/logos/by}
\includegraphics[width=.9cm]{\olpath/assets/logos/cc}
\includegraphics[width=.9cm]{\olpath/assets/logos/remix}
\end{minipage}
\hspace{.3cm}
\begin{minipage}[b]{6.5cm}
\ollicensefont
\textit{Logical Theory} is licensed under a
\href{http://creativecommons.org/licenses/by/4.0/}{Creative Commons
  Attribution 4.0 International License}. It is based on
\textit{\href{https://github.com/OpenLogicProject/OpenLogic}{The Open
    Logic Text}} by the \href{http://openlogicproject.org/}{Open Logic
  Project}, used under a
\href{http://creativecommons.org/licenses/by/4.0/}{Creative Commons
  Attribution 4.0 International License}, and \textit{\href{http://people.ds.cam.ac.uk/tecb2/metatheory.shtml}{Metatheory}} by Tim Button, also under a \href{http://creativecommons.org/licenses/by/4.0/}{Creative Commons
  Attribution 4.0 International License}.
\end{minipage}
\hfill
\href{http://openlogicproject.org/}{\includegraphics[width=2.7cm]{\olpath/assets/logos/openlogic-logo-bw}}


\end{titlingpage}

\frontmatter
\pagestyle{ruled}

%\newpage
\thispagestyle{empty}%
\noindent This text is a remix of the \textit{Open Logic Text} tailor-made for the course Logical theory, LOG110, at the University of Gothenburg. The original text as well as the present text are released under a \href{http://creativecommons.org/licenses/by/4.0/}{Creative Commons
  Attribution 4.0 International license}. Please see \href{http://openlogicproject.org/}{openlogicproject.org} for
more information.

Some modified parts from Tim Button's book \textit{Metatheory} are also included in this text. \textit{Metatheory} is generously released under a Creative Commons license making it possible to include parts of it here.

This version of the text was compiled on \today. Please check the Canvas activity of the course for the most recent version.
If you find typos, errors or have suggestions for improvement please contact your course instructor.

\tableofcontents*

\mainmatter


% PART: PROPOSITIONAL LOGIC

{\let\newpage\relax
\let\clearpage\relax
\part{Propositional Logic}
\chapter{Syntax and Semantics}}

\olimport*[propositional-logic/syntax-and-semantics]{introduction}
\olimport*[propositional-logic/syntax-and-semantics]{formulas}
\olimport*[propositional-logic/syntax-and-semantics]{preliminaries}
\olimport*[propositional-logic/syntax-and-semantics]{valuations-sat}
\olimport*[propositional-logic/syntax-and-semantics]{semantic-notions}
\olimport*[propositional-logic/syntax-and-semantics]{normal-form}
\olimport*[propositional-logic/syntax-and-semantics]{expressive-adequacy}

\OLEndChapterHook

\tagfalse{FOL}
\settexttoken{sentence}{formula}{formulas}
\chapter{Natural Deduction}
\olimport*[first-order-logic/proof-systems]{introduction}
\olimport*[first-order-logic/proof-systems]{natural-deduction}
\olimport*[first-order-logic/natural-deduction]{rules-and-proofs}
\olimport*[first-order-logic/natural-deduction]{propositional-rules}
\olimport*[first-order-logic/natural-deduction]{derivations}
\olimport*[first-order-logic/natural-deduction]{proving-things}
\olimport*[first-order-logic/natural-deduction]{proof-theoretic-notions}
\olimport*[first-order-logic/natural-deduction]{provability-consistency}
\olimport*[first-order-logic/natural-deduction]{provability-propositional}
\olimport*[first-order-logic/natural-deduction]{soundness}
\OLEndChapterHook

%\olimport*[first-order-logic/natural-deduction]{natural-deduction}
\olimport*[first-order-logic/completeness]{completeness}
\tagtrue{FOL}
\settexttoken{sentence}{sentence}{sentences}

\OLEndPartHook

% PART: FIRST-ORDER LOGIC


\cleardoublepage
{\let\newpage\relax
\let\clearpage\relax
\part{First-order Logic}
\chapter{Syntax and Semantics}}
\olimport*[first-order-logic/syntax-and-semantics]{introduction}
\olimport*[first-order-logic/syntax-and-semantics]{first-order-languages}
\olimport*[first-order-logic/syntax-and-semantics]{terms-formulas}
\olimport*[first-order-logic/syntax-and-semantics]{unique-readability}
\olimport*[first-order-logic/syntax-and-semantics]{main-operator}
\olimport*[first-order-logic/syntax-and-semantics]{subformulas}
\olimport*[first-order-logic/syntax-and-semantics]{free-vars-sentences}
\olimport*[first-order-logic/syntax-and-semantics]{substitution}
\olimport*[first-order-logic/syntax-and-semantics]{structures}
\olimport*[first-order-logic/syntax-and-semantics]{covered-structures}
\olimport*[first-order-logic/syntax-and-semantics]{satisfaction}
\olimport*[first-order-logic/syntax-and-semantics]{assignments}
\olimport*[first-order-logic/syntax-and-semantics]{extensionality}
\olimport*[first-order-logic/syntax-and-semantics]{semantic-notions}
\OLEndChapterHook

%\olimport*[first-order-logic/syntax-and-semantics]{syntax-and-semantics}
\olimport*[first-order-logic/models-theories]{models-theories}

\chapter{Natural Deduction}

\olsection{Introduction}

To define a derivation system for first-order logic we will use what we already have for propositional logic and add rules for the quantifiers.

%This needs to be expanded.


%\olimport*[first-order-logic/proof-systems]{introduction}
%\olimport*[first-order-logic/proof-systems]{natural-deduction}
%\olimport*[first-order-logic/natural-deduction]{rules-and-proofs}
%\olimport*[first-order-logic/natural-deduction]{propositional-rules}
%\olimport*[first-order-logic/natural-deduction]{derivations}
%\olimport*[first-order-logic/natural-deduction]{proving-things}
\olimport*[first-order-logic/natural-deduction]{quantifier-rules}
\olimport*[first-order-logic/natural-deduction]{proving-things-quant}
\olimport*[first-order-logic/natural-deduction]{proof-theoretic-notions}
\olimport*[first-order-logic/natural-deduction]{provability-consistency}
\olimport*[first-order-logic/natural-deduction]{provability-propositional}
\olimport*[first-order-logic/natural-deduction]{provability-quantifiers}
\olimport*[first-order-logic/natural-deduction]{soundness}
\olimport*[first-order-logic/natural-deduction]{identity}
\olimport*[first-order-logic/natural-deduction]{soundness-identity}
\OLEndChapterHook

\olimport*[first-order-logic/completeness]{completeness}
%\olimport*[first-order-logic/beyond]{beyond}

%\OLEndPartHook

% PART: MODEL THEORY

\cleardoublepage
{\let\newpage\relax
\let\clearpage\relax
%\part{Some Model Theory}

\chapter{Basics of Model Theory}}

\olimport*[model-theory/basics]{reducts-and-expansions}
\olimport*[model-theory/basics]{substructures}
\olimport*[model-theory/basics]{overspill}
\olimport*[model-theory/basics]{isomorphism}
\olimport*[model-theory/basics]{theory-of-m}
%\olimport{partial-iso}
%\olimport{dlo}
%\olimport{nonstandard-arithmetic}
%\OLEndChapterHook

%\chapter{Models of Arithmetic}
\section{Models of Arithmetic}

The \emph{standard model} of aritmetic is the
!!{structure}~$\Struct{N}$ with $\Domain{N} = \Nat$ in which
$\Obj{0}$, $\prime$, $+$, $\times$, and $<$ are interpreted as you
would expect. That is, $\Obj{0}$ is $0$, $\prime$ is the successor
function, $+$ is interpeted as addition and $\times$ as multiplication
of the numbers in~$\Nat$. Specifically,
\begin{align*}
  \Assign{\Obj{0}}{N} & = 0\\
  \Assign{\prime}{N}(n) & = n + 1\\
  \Assign{+}{N}(n, m) & = n + m\\
  \Assign{\times}{N}(n, m) & = nm
\end{align*}
Of course, there are structures for $\Lang{L_A}$ that have domains
other than~$\Nat$. For instance, we can take $\Struct{M}$ with domain
$\Domain{M} = \{a\}^*$ (the finite sequences of the single
symbol~$a$, i.e., $\emptyset$, $a$, $aa$, $aaa$, \dots), and
interpretations
\begin{align*}
  \Assign{\Obj{0}}{M} & = \emptyset\\
  \Assign{\prime}{M}(s) & = s \concat a\\
  \Assign{+}{M}(n, m) & = a^{n + m}\\
  \Assign{\times}{M}(n, m) & = a^{nm}
\end{align*}
These two structures are ``essentially the same'' in the sense that
the only difference is the !!{element}s of the !!{domain}s but not how
the !!{element}s of the !!{domain}s are related among each other by
the interpretation functions. We say that the two !!{structure}s are
\emph{isomorphic}.

It is an easy consequence of the compactness theorem that any theory
true in~$\Struct{N}$ also has models that are not isomorphic
to~$\Struct{N}$.  Such structures are called \emph{non-standard}.  The
interesting thing about them is that while the !!{element}s of a
standard model (i.e., $\Struct{N}$, but also all !!{structure}s
isomorphic to it) are exhausted by the values of the standard
numerals~$\num{n}$, i.e.,
\[
\Domain{N} = \Setabs{\Value{\num{n}}{N}}{n \in \Nat}
\]
that isn't the case in non-standard models: if $\Struct{M}$ is
non-standard, then there is at least one $x \in \Domain{M}$ such that
$x \neq \Value{\num{n}}{M}$ for all~$n$.

\begin{defn}
The theory of \emph{true arithmetic} is the set of !!{sentence}s
satisfied in the standard model of arithmetic, i.e.,
\[
\Th{TA} = \Setabs{!A}{\Sat{N}{!A}}.
\]
\end{defn}

\begin{defn}
The theory $\Th{Q}$ axiomatized by the following sentences is known
as ``Robinson's $\Th{Q}$'' and is a very simple theory of arithmetic.
\begin{align*}
& \lforall[x][\lforall[y][(\eq[x'][y'] \lif \eq[x][y])]] \tag{$!Q_1$}\\
& \lforall[x][\eq/[\Obj 0][x']] \tag{$!Q_2$}\\
& \lforall[x][(\eq/[x][\Obj 0] \lif \lexists[y][\eq[x][y']])] \tag{$!Q_3$}\\
& \lforall[x][\eq[(x + \Obj 0)][x]] \tag{$!Q_4$}\\
& \lforall[x][\lforall[y][\eq[(x + y')][(x + y)']]] \tag{$!Q_5$}\\
& \lforall[x][\eq[(x \times \Obj 0)][\Obj 0]] \tag{$!Q_6$}\\
& \lforall[x][\lforall[y][\eq[(x \times y')][((x \times y) + x)]]] \tag{$!Q_7$}\\
& \lforall[x][\lforall[y][(x < y \liff \lexists[z][\eq[(z' + x)][y]])]] \tag{$!Q_8$}
\end{align*}
The set of !!{sentence}s $\{!Q_1, \dots, !Q_8\}$ are the axioms of
$\Th{Q}$, so $\Th{Q}$ consists of all !!{sentence}s entailed by them:
\[
\Th{Q} = \Setabs{!A}{\{!Q_1, \dots, !Q_8\} \Entails !A}.
\]
\end{defn}

\begin{defn}
Suppose $!A(x)$ is !!a{formula} in $\Lang L_A$ with free variables~$x$
and $y_1$, \dots, $y_n$. Then any !!{sentence} of the form
\[
\lforall[y_1][\dots\lforall[y_n][((!A(\Obj 0) \land \lforall[x][(!A(x)
\lif !A(x'))]) \lif \lforall[x][!A(x)])]]
\]
is an instance of the \emph{induction schema}.

\emph{Peano arithmetic}~$\Th{PA}$ is the theory axiomatized by the
axioms of $\Th{Q}$ together with all instances of the induction
schema.
\end{defn}


%\olimport*[model-theory/models-of-arithmetic]{introduction}
\olimport*[model-theory/models-of-arithmetic]{standard-models}
\olimport*[model-theory/models-of-arithmetic]{non-standard-models}
%\olimport{models-of-q}
%\olimport{computable-models}
\OLEndChapterHook

%\olimport[interpolation]{interpolation}
%\olimport[lindstrom]{lindstrom}
\OLEndPartHook


% PART: SECOND-ORDER LOGIC

\cleardoublepage
{\let\newpage\relax
\let\clearpage\relax
\part{Second-order Logic}
\chapter{Syntax and Semantics}}

\olimport*[second-order-logic/syntax-and-semantics]{introduction}
\olimport*[second-order-logic/syntax-and-semantics]{terms-formulas}
\olimport*[second-order-logic/syntax-and-semantics]{satisfaction}
\olimport*[second-order-logic/syntax-and-semantics]{semantic-notions}
\olimport*[second-order-logic/syntax-and-semantics]{expressive-power}
\olimport*[second-order-logic/syntax-and-semantics]{inf-count}
%\OLEndChapterHook
%\olimport*[second-order-logic/metatheory]{metatheory}
%\chapter{Metatheory of Second-order Logic}
%\olimport*[second-order-logic/metatheory]{introduction}
%\olimport*[second-order-logic/metatheory]{undecidability-and-axiomatizability}
\olimport*[second-order-logic/metatheory]{compactness}
\olimport*[second-order-logic/metatheory]{loewenheim-skolem}
\olimport*[second-order-logic/metatheory]{second-order-arithmetic}
\OLEndChapterHook

%\olimport*[second-order-logic/sol-and-set-theory]{sol-and-set-theory}
%\chapter{Beyond Second-order}
%\olimport*[first-order-logic/beyond]{higher-order-logic}
%\OLEndChapterHook

\OLEndPartHook

% PART: INTUITIONISTIC LOGIC

\cleardoublepage
{\let\newpage\relax
\let\clearpage\relax
\part{Intuitionistic Logic}
\chapter{Introduction}}

\olimport*[intuitionistic-logic/introduction]{constructive-reasoning}
\olimport*[intuitionistic-logic/introduction]{syntax}
\olimport*[intuitionistic-logic/introduction]{bhk-interpretation}
\olimport*[intuitionistic-logic/introduction]{natural-deduction}
%\olimport*[intuitionistic-logic/introduction]{axiomatic-derivations}
\OLEndChapterHook

%\olimport*[intuitionistic-logic/semantics]{semantics}
\chapter{Semantics}
\olimport*[intuitionistic-logic/semantics]{introduction}
\olimport*[intuitionistic-logic/semantics]{relational-models}
\olimport*[intuitionistic-logic/semantics]{semantic-notions}
%\olimport*[intuitionistic-logic/semantics]{topological-semantics}
\OLEndChapterHook

%\olimport*[intuitionistic-logic/soundness-completeness]{soundness-completeness}
\chapter{Soundness and Completeness}
%\olimport*[intuitionistic-logic/soundness-completeness]{soundness-axd}
\olimport*[intuitionistic-logic/soundness-completeness]{soundness-nd}
\olimport*[intuitionistic-logic/soundness-completeness]{lindenbaum}
\olimport*[intuitionistic-logic/soundness-completeness]{canonical-model}
\olimport*[intuitionistic-logic/soundness-completeness]{truth-lemma}
\olimport*[intuitionistic-logic/soundness-completeness]{completeness-thm}
\OLEndChapterHook

%\olimport*[intuitionistic-logic/propositions-as-types]{propositions-as-types}

\OLEndPartHook


% PART: COMPUTABILITY AND INCOMPLETENESS

\cleardoublepage
{\let\newpage\relax
\let\clearpage\relax
\part{Computability and Incompleteness}


\chapter{Turing Machine Computations}}

\olimport*[turing-machines/machines-computations]{introduction}
\olimport*[turing-machines/machines-computations]{representing-tms}
\olimport*[turing-machines/machines-computations]{turing-machines}
\olimport*[turing-machines/machines-computations]{configuration}
\olimport*[turing-machines/machines-computations]{unary-numbers}
\olimport*[turing-machines/machines-computations]{halting-states}
\olimport*[turing-machines/machines-computations]{disciplined-machines}
\olimport*[turing-machines/machines-computations]{combining-machines}
\olimport*[turing-machines/machines-computations]{variants}
\olimport*[turing-machines/machines-computations]{church-turing-thesis}
\OLEndChapterHook

\olimport*[turing-machines/undecidability]{undecidability}


\chapter{Recursive Functions}

\olimport*[computability/recursive-functions]{introduction}
\olimport*[computability/recursive-functions]{primitive-recursion}
\olimport*[computability/recursive-functions]{composition}
\olimport*[computability/recursive-functions]{pr-functions}
\olimport*[computability/recursive-functions]{notation-pr-functions}
\olimport*[computability/recursive-functions]{pr-functions-computable}
\olimport*[computability/recursive-functions]{examples}
\olimport*[computability/recursive-functions]{pr-relations}
\olimport*[computability/recursive-functions]{bounded-minimization}
\olimport*[computability/recursive-functions]{primes}
\olimport*[computability/recursive-functions]{sequences}
\olimport*[computability/recursive-functions]{trees}
\olimport*[computability/recursive-functions]{other-recursions}
\olimport*[computability/recursive-functions]{non-pr-functions}
\olimport*[computability/recursive-functions]{partial-functions}
%\olimport*[computability/recursive-functions]{normal-form}
%\olimport*[computability/recursive-functions]{halting-problem}
\olimport*[computability/recursive-functions]{general-recursive-functions}
\OLEndChapterHook

\olimport*[incompleteness/introduction]{introduction}

\olimport*[incompleteness/arithmetization-syntax]{arithmetization-syntax}

\olchapter{inc}{req}{Representability in \texorpdfstring{$\Th{Q}$}{Q}}
\olimport*[incompleteness/representability-in-q]{introduction}
\olimport*[incompleteness/representability-in-q]{representable-comp}
\olimport*[incompleteness/representability-in-q]{beta-function}
\olimport*[incompleteness/representability-in-q]{prim-rec}
\olimport*[incompleteness/representability-in-q]{basic-representable}
\olimport*[incompleteness/representability-in-q]{composition-representable}
\olimport*[incompleteness/representability-in-q]{minimization-representable}
\olimport*[incompleteness/representability-in-q]{comp-representable}
\olimport*[incompleteness/representability-in-q]{representing-relations}
%\olimport*[incompleteness/representability-in-q]{undecidability}
\OLEndChapterHook

% leave out this part -- it depends on computability theory chapter
% \olimport[incompleteness/theories-computability]{theories-computability}
%\olimport*[incompleteness/incompleteness-provability]{incompleteness-provability}




\olchapter{inc}{inp}{Incompleteness and Provability}
\olimport*[incompleteness/incompleteness-provability]{introduction}
\olimport*[incompleteness/incompleteness-provability]{fixed-point-lemma}
\olimport*[incompleteness/incompleteness-provability]{first-incompleteness-thm}
\olimport*[incompleteness/incompleteness-provability]{rosser-thm}
\olimport*[incompleteness/incompleteness-provability]{godels-paper}
%\olimport*[incompleteness/incompleteness-provability]{provability-conditions}
%\olimport*[incompleteness/incompleteness-provability]{second-incompleteness-thm}
%\olimport*[incompleteness/incompleteness-provability]{lob-thm}
%\olimport*[incompleteness/incompleteness-provability]{tarski-thm}
\OLEndChapterHook



\OLEndPartHook


\appendix

\part{Appendices}

\olimport*[methods/proofs]{proofs}
\olimport*[methods/induction]{induction}
\olimport*[history/biographies]{biographies}

\OLEndPartHook

\backmatter

\photocredits

\bibliographystyle{\olpath/bib/natbib-oup}
\bibliography{\olpath/bib/open-logic}

\end{document}

